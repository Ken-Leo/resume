%%
%% Copyright (c) 2018-2019 Ke Luo <kenleo_lucas@outlook.com>
%% CC BY 4.0 License
%%
%% Created: 2018-04-11
%%

% Chinese version
\documentclass[zh]{resume}
% Format with gb7714-2015 
\usepackage[backend=biber,style=gb7714-2015,maxnames=12]{biblatex}
\addbibresource[location=local]{resume-zh.bib}

% Highlight the author and the corresponding author
\makeatletter
\renewcommand*{\mkbibnamegiven}[1]{%
\ifitemannotation{theauthor}
{\ifbibliography{\textbf{\textbf{#1}}}{#1}}%
{#1}\ifbibliography{\ifitemannotation{corresponding}{\textsuperscript{*}}{}}{}%
}

\renewcommand*{\mkbibnamefamily}[1]{%
\ifitemannotation{theauthor}
{\ifbibliography{\textbf{\textbf{#1}}}{#1}}
{#1}
}
\makeatother




% Adjust icon size (default: same size as the text)
\iconsize{\Large}

% File information shown at the footer of the last page
\fileinfo{%
  \faCopyright{} 2016--2025, Ke Luo \hspace{0.5em}
  \creativecommons{by}{4.0} \hspace{0.5em}
  \githublink{Ken-Leo}{resume} \hspace{0.5em}
  \faEdit{} \today
}

\name{可}{罗}

\keywords{Storage, Magnetic recording, HDD, Tape, Optical storage, Signal processing, ECC, Python, C, Matlab/Octave}

% \tagline{\icon{\faBinoculars}} <position-to-look-for>}
% \tagline{<current-position>}

% \photo{<height>}{<filename>}

\profile{
  % \mobile{(+86)131-3567-8223}
  \email{kenleo\_lucas\#outlook\#com}
  \github{Ken-Leo} \\
  \university{华中科技大学}
  \degree{计算机系统结构 \textbullet 博士}
  \birthday{1993-01-02}
  \address{武汉}
  % Custom information:
  % \icontext{<icon>}{<text>}
  % \iconlink{<icon>}{<link>}{<text>}
}

\begin{document}
\makeheader
%======================================================================
% Summary & Objectives
%======================================================================
{\onehalfspacing\hspace{2em}%
计算机系统结构专业博士,
擅长磁存储及光存储系统建模与分析,热衷数据存储技术、信号处理与信息理论。
\par}

% %======================================================================
% \sectionTitle{技能和语言}{\faWrench}
% %======================================================================
% \begin{competences}
%   \comptence{操作系统}{%
%     \icon{\faLinux} Linux (10 年),
%     \icon{\faFreebsd} DragonFly BSD \& FreeBSD (7 年)
%   }
%   \comptence{编程}{%
%     Python, C, Shell, R, Tcl/Tk
%   }
%   \comptence{工具}{%
%     SSH, Git, Make, Tmux, Vi, Ansible
%   }
%   \comptence{数据分析}{%
%     R, Pandas; Matplotlib, ggplot2; Keras, Scikit-learn
%   }
%   \comptence{网站开发}{%
%     Flask, JavaScript, jQuery, Bootstrap
%   }
%   \comptence{\icon{\faLanguage} 语言}{
%     \textbf{英语} --- 读写(优良),听说(日常交流)
%   }
% \end{competences}

%======================================================================
\sectionTitle{教育背景}{\faGraduationCap}
%======================================================================
\begin{educations}
  \education%
    {2023.03}%
    [至今]%
    {华中科技大学}%
    {武汉光电国家研究中心}%
    {光学工程}%
    {博士后}%
    {合作导师:陈进才,张静宇}

  \separator{0.5ex}
  \education%
    {2016.09}%
    [2022.12]%
    {华中科技大学}%
    {武汉光电国家研究中心}%
    {计算机系统结构}%
    {博士}%
    {导师:陈进才}

  \separator{0.5ex}
  \education%
    {2012.09}%
    [2016.06]%
    {中南民族大学}%
    {电子信息工程学院}%
    {电子信息工程}%
    {学士}%
    {}
\end{educations}

% %======================================================================
% \sectionTitle{计算机技能}{\faCogs}
% %======================================================================
% \begin{itemize}
%   \item DragonFly BSD 操作系统开发者:
%     200+ 代码提交;内核以及系统工具;
%     在邮件列表和 IRC 频道交流和回答问题
%   \item 使用 Ansible 管理 VPS,部署个人域名邮箱、权威 DNS、网站、Git、IRC 等服务
%   \item 搭建并管理课题组的工作站、计算集群(4 节点)和网络设备
%   \item 参与配置和测试上海天文台的 SKA 高性能计算集群原型机
%     (1 管理节点 + 1 存储节点 + 4 计算节点)
%   \item 设计并开发了\enquote{2014 第一届中国—新西兰联合 SKA 暑期学校}的整个网站
%     (Django, Bootstrap, jQuery)
% \end{itemize}

%======================================================================
\sectionTitle{科研项目}{\faCode}
%======================================================================
\begin{itemize}
  \item 国家自然科学基金,面上项目,62272178, 超高密度三维热辅助磁记录写机制研究,2023/01至2026/12,在研,参与
  \begin{itemize}
    \item 研究成果与创新点:三维磁记录介质模型,三维记录数据回读模型。
    \item 主要贡献: 申请书的撰写并获得资助,三维磁记录读通道建模,三维回读信号均衡检测。
    \item 研究成果影响:实现双层磁数据记录保障三维磁记录数据读取可靠性以达到存储密度倍增,为超高密度磁存储提供理论基础和技术支持。
  \end{itemize}

  \item 国家自然科学基金,面上项目,61672246,超高密度二维磁记录读磁头阵列及其记录系统关键技术研究,2017/01至2020/12,已结题,参与
  \begin{itemize}
    \item 研究成果与创新点:二维磁记录介质模型,二维记录数据回读模型。
    \item 主要贡献:开展二维磁记录介质建模、读写过程建模及回读信号处理,设计了二维 Voronoi 介质颗粒模型的生成方法,进行了不同介质及记录位尺寸参数下的写错误分析。
    \item 研究成果影响:提出了平衡准确性和计算开销的二维回读响应区间的选取指标,搭建了二维磁记录写入、回读和数据恢复信道的系统结构,提出了基于神经网络的块均衡检测方法、限制连续磁化跃迁的约束控制编码方法等,有效提高了磁存储系统的可靠性,对大幅提升磁存储密度和容量具有重要的理论指导意义与应用价值。
  \end{itemize}

  \item 国家自然科学基金,面上项目,61272068,比特图案介质的超高密度瓦记录关键技术研究,2013/01-2016/12,已结题,参与
  \begin{itemize}
    \item 研究成果与创新点:比特图案磁记录介质模型,比特图案记录数据回读模型。
    \item 主要贡献:比特图案磁记录读通道建模,记录位排列方式研究,回读信号仿真分析等。
    \item 研究成果影响:提出了一种基于比特图案的磁记录介质模型,实现了基于该模型的磁记录读通道建模,为比特图案介质的瓦记录技术提供理论基础和技术支持。
  \end{itemize}
  
  \item 企业横向,高密度HAMR信号处理技术合作项目,2025/08至2026/06,在研,主持
  % \begin{itemize}
  %   \item 研究成果与创新点:基于独立引导层的多层蓝光光盘PRML信号处理算法与仿真模型。
  %   \item 主要贡献:多层光盘PRML模型研究及信号质量评估方法研究。
  %   \item 研究成果影响:300GB-500GB多层蓝光光盘实现RF信号样本的检测,达到商用要求,为超大容量光盘的国产化提供技术支持。
  % \end{itemize}

  \item 企业横向,面向蓝光超多层PRML算法技术合作项目,2024/07至2025/02,在研,参与
  \begin{itemize}
    \item 研究成果与创新点:基于独立引导层的多层蓝光光盘PRML信号处理算法与仿真模型。
    \item 主要贡献:多层光盘PRML模型研究及信号质量评估方法研究。
    \item 研究成果影响:300GB-500GB多层蓝光光盘实现RF信号样本的检测,达到商用要求,为超大容量光盘的国产化提供技术支持。
  \end{itemize}

  \item 企业横向,基于BDXL标准的PRML模型设计与实现合作项目,2022/08至2023/06,已结题,参与
  \begin{itemize}
    \item 研究成果与创新点:蓝光光盘BDXL标准的PRML检测器设计。
    \item 主要贡献:完成一套基于BDXL标准的蓝光存储PRML仿真模型的算法设计。
    \item 研究成果影响:所设计的PRML通道方案在商用光盘中测试通过,检测效果达到国际同类先进水平。
  \end{itemize}

  \item 企业横向,HDD 原型算法和先进磁记录技术合作项目,2022/03至2023/03,已结题,参与
  \begin{itemize}
    \item 研究成果与创新点:磁记录系统微磁学仿真分析及建模,先进磁记录系统算法设计和仿真。
    \item 主要贡献:开展基于PMR/TDMR的HDD相关技术、算法和文献调研,进行PMR、TDMR优化算法选型研究,开展磁头、存储密度、可靠性、均衡、检测、编译码等方面关键技术的开发、设计和优化。
    \item 研究成果影响:基于PMR+TDMR磁记录技术,提供原型/商用可实现、性能在业界有竞争力的HDD算法以及浮点算法仿真代码开发,支撑原型系统开发和验证。
  \end{itemize}
\end{itemize}

%======================================================================
\sectionTitle{科研成果}{\faAtom}
%======================================================================
% \begin{itemize}
%   \item 开发低频射电天空图像模拟软件:
%     \link{https://github.com/liweitianux/fg21sim}{\texttt{FG21sim}}
%   \item 开发程序实现~X~射线天文观测数据的半自动化分析:
%     \link{https://github.com/liweitianux/chandra-acis-analysis}{\texttt{chandra-acis-analysis}}
%   \item 利用卷积去噪自动编码器(CDAE)在频率维度分离微弱的宇宙再电离(EoR)信号
%   \item 利用卷积神经网络(CNN)对 FIRST 巡天的射电星系图像根据形态特征进行分类
%   \item 显著改进星系团射电晕的建模,并考虑低频干涉阵列的复杂仪器效应
%   \item 改进~X~射线光谱拟合的背景成分建模,获到更准确可靠的拟合结果
%   \item 发表 2 篇第一作者以及 8 篇合作者 SCI 论文
% \end{itemize}

\onehalfspacing\hspace{2em}%
\nocite{*}
% \printbibliography[heading=bibliography,title=论文及专利]
\printbibliography[heading={none}]
\par

%% Format with ieeetrans style
% \bibliographystyle{IEEEtran}{
%   \onehalfspacing\hspace{2em}%
%   \nocite{*}
%   \bibliography{resume-zh}
%   \par
% }

% %======================================================================
% \sectionTitle{实习经历}{\faBriefcase}
% %======================================================================
% \begin{experiences}
%   \experience%
%     [2018.04]%
%     {2018.08}%
%     {数据工程师 @ 上海领脉网络科技(初创公司)}%
%     [\begin{itemize}
%       \item 从 Amazon 网页搜索并挖取商品与广告信息
%         (Python, Requests, BeautifulSoup)
%       \item 配置 Airflow 服务器和数据库等基础设施,
%         定期从 Amazon 获取产品销售与广告投放等数据
%       \item 开发网站(Flask, jQuery),帮助客户优化 Amazon 广告投放
%     \end{itemize}]

%   \separator{0.5ex}
%   \experience%
%     [2013.07]%
%     {2013.09}%
%     {网站开发 @ 97 随访(初创公司)}%
%     [\begin{itemize}
%       \item 后端开发(Django),完成用户注册、数据存储和搜索等功能
%       \item 前端开发(jQuery, AJAX),对患者各项指标随时间的变化进行可视化
%     \end{itemize}]
% \end{experiences}

\end{document}
