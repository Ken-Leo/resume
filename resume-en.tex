%%
%% Copyright (c) 2018-2019 Weitian LI <wt@liwt.net>
%% CC BY 4.0 License
%%
%% Résumé
%% ------
%% A short document (1-2 pages) to sum up the job-related accomplishments
%% and experience.
%%
%% Checklist
%% ---------
%% * Contact Information
%% * Work History / Experience
%% * Education
%% * Skills
%% * Summary & Objective (optional)
%% * Hobbies & Interests (optional)
%%
%% Credits
%% -------
%% * CV vs. Resume: What is the Difference? When to Use Which?
%%   https://uptowork.com/blog/cv-vs-resume-difference
%% * How to Make a Resume: A Step-by-Step Guide (+30 Examples)
%%   https://uptowork.com/blog/how-to-make-a-resume
%% * Entry-Level Resume: Sample and Complete Guide (+20 Examples)
%%   https://uptowork.com/blog/entry-level-resume-example
%%
%% Created: 2018-04-14
%%

% English version
\documentclass[en]{resume}

%%
%% Bibliography
%%
\usepackage[backend=bibtex,style=ieee,maxnames=20, giveninits=false, uniquename=false]{biblatex} 
\addbibresource[location=local]{resume-en.bib}

%%
%% Highlight the author and the corresponding author
%%
% Have not found a way to highlight correctly.

\renewcommand*{\mkbibnamefamily}[1]{%
  \ifthenelse{\equal{#1}{Luo\textsuperscript{*}}\OR\equal{#1}{Luo}}{\textbf{#1}}{#1}%
}

\renewcommand*{\mkbibnamegiven}[1]{%
\ifthenelse{\equal{#1}{Ke}}{\textbf{#1}}{#1}%
}

% Adjust icon size (default: same size as the text)
\iconsize{\Large}

% File information shown at the footer of the last page
\fileinfo{%
  \faCopyright{} 2016--2025, Ke Luo@HUST \hspace{0.5em}
  \creativecommons{by}{4.0} \hspace{0.5em}
  \githublink{Ken-Leo}{resume} \hspace{0.5em}
  \faEdit{} \today
}

\name{Ke}{Luo}

\keywords{Storage, Magnetic recording, HDD, Tape, Optical storage, Signal processing, ECC, Python, C, Matlab/Octave}

% \tagline{\icon{\faBinoculars}} <position-to-look-for>}
% \tagline{<current-position>}

% \photo{<height>}{<filename>}

\profile{
  % \mobile{(+86)131-3567-8223}
  % \email{luoke\_kenleo\#hust\#edu\#cn}
  \email{luoke\_kenleo@hust.edu.cn}
  \github{Ken-Leo} \\
  \degree{Ph.D. in Computer System Architecture}
  \university{Huazhong University of Science and Technology (HUST)}
  \birthday{1993 Jan.}
  \address{Wuhan}
  % Custom information:
  % \icontext{<icon>}{<text>}
  % \iconlink{<icon>}{<link>}{<text>}
}


\begin{document}
\makeheader

%======================================================================
% Summary & Objectives
%======================================================================
Highly-motivated Ph.D. in Computer Science(Computer Architecture)
with good foundations of math and statistics.
Proficient in storage channel modeling, analysis, and signal processing
and enthusiastic about data storage technologies and deep learning inspired information theory.
Skilled in Matlab/Octave, Python, and C/C++ programming.
Passionate about computer science, hiking, and photography.

% %======================================================================
% \sectionTitle{Competences \& Languages}{\faWrench}
% %======================================================================
% \begin{competences}[10em]
%   \comptence{Operating Systems}{
%     \icon{\faLinux} Linux (10 years),
%     \icon{\faFreebsd} DragonFly BSD \& FreeBSD (7 years)
%   }
%   \comptence{Programming}{%
%     Python, C, Shell, R, Tcl/Tk
%   }
%   \comptence{Tools}{%
%     SSH, Git, Make, Tmux, Vi, Ansible
%   }
%   \comptence{Data Analysis}{%
%     R, Pandas; Matplotlib, ggplot2; Keras, Scikit-learn
%   }
%   \comptence{Web Development}{%
%     Flask, JavaScript, jQuery, Bootstrap
%   }
%   \comptence{\icon{\faLanguage} Languages}{
%     \textbf{English} ---
%       reading \& writing (good);
%       listening \& speaking (conversant)
%   }
% \end{competences}

%======================================================================
\sectionTitle{Education}{\faGraduationCap}
%======================================================================
\begin{educations}
  \education%
  {March 2023}%
    [Till now]%
    {Huazhong University of Science and Technology}%
    {Wuhan National Laboratory for Optoelectronics}%
    {Optical Engineering}%
    {Postdoc.}%
    {, Mentor: Prof.~Jincai Chen, Prof.~Jingyu Zhang}%

  \separator{0.5ex}
  \education%
    {September 2016}%
    [December 2022]%
    {Huazhong University of Science and Technology}%
    {Wuhan National Laboratory for Optoelectronics}%
    {Computer Architecture}%
    {Ph.D.}%
    {, Supervisor: Prof.~Jincai Chen}%

    \separator{0.5ex}
  \education%
    {September 2012}%
    [June 2016]%
    {South-Central Minzu University}%
    {College of Electronics and Information}%
    {Electronics and Information Engineering}%
    {Bechelor}%
    {}%

\end{educations}

% %======================================================================
% \sectionTitle{Computer Skills}{\faCogs}
% %======================================================================
% \begin{itemize}
%   \item DragonFly BSD operating system developer:
%     200+ code commits; kernel and system utilities;
%     participate in discussions and anwser questions
%     in mailing lists and the IRC channel.
%   \item Use Ansible to manage a VPS running DragonFly BSD that serves
%     personal email, authoritative DNS, website, Git, IRC, etc.
%   \item Built and administrate the workstations, a 4-node computer cluster,
%     and network facilities for the team.
%   \item Participated in building and testing the SKA high-performance
%     cluster prototype (1 login node + 1 data node + 4 computing nodes)
%     in Shanghai Astronomical Observatory.
%   \item Designed and developed the whole website (Django, Bootstrap, jQuery)
%     for \enquote{The 1st China--New Zealand Joint SKA Summer School}
%     in 2014.
% \end{itemize}

%======================================================================
\sectionTitle{Research Projects}{\faCode}
%======================================================================
\begin{itemize}
  \item National Natural Science Foundation of China, General Program, No. 62272178, 
  Study of Write Mechanism of Ultra-high Density Three-Dimensional Heat-Assisted Magnetic Recording, 2023/01 to 2026/12, ongoing, participant
  \begin{itemize}
    \item Research Achievements and Innovations: 3D magnetic recording medium model, 3D data readback model.
    \item Main Contributions: Writing the proposal and obtaining funding, 3D magnetic recording read channel modeling, 3D readback signal equalization and detection.
    \item Impact of Research Achievements: Achieved dual-layer magnetic data recording to ensure the reliability of 3D magnetic data reading, doubling the storage density, providing theoretical foundation and technical support for ultra-high-density 3D magnetic storage.
  \end{itemize}

  \item National Natural Science Foundation of China, General Program, No. 61672246, Key Technologies of Read Heads Array and The Recording System for
  Two-Dimensional Magnetic Recording at Ultra-high Densities, 2017/01 to 2020/12, completed, participant
  \begin{itemize}
    \item Research Achievements and Innovations: 2D magnetic recording medium model, 2D data readback model.
    \item Main Contributions: Conducted modeling of 2D magnetic recording media, read/write processes, and readback signal processing. Designed a method for generating a 2D Voronoi medium particle model and analyzed write errors under different medium and recording bit size parameters.
    \item Impact of Research Achievements: Proposed an index for selecting the 2D readback response interval that balances accuracy and computational overhead. Built a system structure for 2D magnetic recording writing, readback, and data recovery channels. Proposed methods such as block equalization detection based on neural networks and constraint-controlled coding methods to limit continuous magnetization transitions, providing theoretical foundation and technical support for ultra-low-density magnetic storage.
  \end{itemize}

  \item National Natural Science Foundation of China, General Program, No. 61272068, Key Technologies of Ultra-high Density Shingled Magnetic Recording on Bit Patterned Media, 2013/01 to 2016/12, completed, participant
  \begin{itemize}
    \item Research Achievements and Innovations: Bit-patterned magnetic recording medium model, bit-patterned data readback model.
    \item Main Contributions: Modeling of bit-patterned magnetic recording read channels, study of recording bit arrangement, and readback signal simulation analysis.
    \item Impact of Research Achievements: Proposed a bit-pattern-based magnetic recording medium model, achieved modeling of the magnetic recording read channel based on this model, and performed write error analysis under different recording bit arrangements, providing theoretical foundation and technical support for ultra-low-density magnetic storage.
  \end{itemize}
  
  \item Corporate Collaboration, PRML Algorithms and Techniques for Multi-layer Blu-ray Discs Cooperation Project, 2024/07 to 2025/02, ongoing, participant
  \begin{itemize}
    \item Research Achievements and Innovations: Multi-layer blue-ray disc PRML signal processing algorithm and simulation model based on an independent guiding layer.
    \item Main Contributions: Research on multi-layer disc PRML models and signal quality assessment methods.
    \item Impact of Research Achievements: Achieved RF signal sample detection for 300GB-500GB multi-layer blue-ray discs, meeting commercial requirements, providing technical support for the domestic production of ultra-large capacity optical discs.
  \end{itemize}

  \item Corporate Collaboration, PRML Model Design and Implementation Based on BDXL Standard, 2022/08 to 2023/06, completed, participant
  \begin{itemize}
    \item Research Achievements and Innovations: PRML detector design for blue-ray discs based on the BDXL standard.
    \item Main Contributions: Completed algorithm design for a PRML simulation model of blue-ray storage based on the BDXL standard.
    \item Impact of Research Achievements: The designed PRML channel solution was tested in commercial optical discs, achieving detection results at the international advanced level.
  \end{itemize}

  \item Corporate Collaboration, HDD Prototype Algorithms and Advanced Magnetic Recording Technology Cooperation Project, 2022/03 to 2023/03, completed, participant
  \begin{itemize}
    \item Research Achievements and Innovations: Micromagnetic simulation analysis and modeling of magnetic recording systems, advanced magnetic recording system algorithm design and simulation.
    \item Main Contributions: Conducted research on HDD-related technologies, algorithms, and literature based on PMR/TDMR, optimized algorithms for PMR and TDMR, and developed, designed, and optimized key technologies related to magnetic heads, storage density, reliability, equalization, detection, and encoding/decoding.
    \item Impact of Research Achievements: Based on PMR+TDMR magnetic recording technology, provided prototype/commercially viable HDD algorithms with competitive performance in the industry, as well as floating-point algorithm simulation code development, supporting prototype system development and verification.
  \end{itemize}
\end{itemize}

%======================================================================
% \sectionTitle{Research Achievements}{\faAtom}
%======================================================================
% \begin{itemize}
%   \item Developed the low-frequency radio sky image simulation software:
%     \link{https://github.com/liweitianux/fg21sim}{\texttt{FG21sim}}.
%   \item Developed a suite of utilities to semi-automate the
%     X-ray astronomical data analysis:
%     \link{https://github.com/liweitianux/chandra-acis-analysis}{\texttt{chandra-acis-analysis}}.
%   \item Separated the faint cosmological EoR signal along the frequency
%     dimension using a Convolutional Denoising Autoencoder (CDAE).
%   \item Classified the radio galaxies in the FIRST survey according to
%     morphologies using a Convolutional Neutral Network (CNN).
%   \item Significantly improved the modeling of radio halos,
%     and integrated the instrumental effects of radio interferometers
%     into the simulation pipeline.
%   \item Improved the background modeling in X-ray spectral fitting
%     achieved more accurate and robust fitting results.
%   \item Published 2 first-author and 8 co-authored SCI papers.
% \end{itemize}


%======================================================================
\sectionTitle{Publications and Patents}{\faAtom}
%======================================================================
% \begin{refsection}
% \bibliographystyle{IEEEtran}
% {\onehalfspacing\hspace{2em}%
% \nocite{*}
% \bibliography{resume-en}
% \par}
% \end{refsection}

% \onehalfspacing\hspace{2em}%
\nocite{*}
\printbibliography[heading={none}]
% \printbibliography[heading=bibliography,title=Publications]
% \par


% %======================================================================
% \sectionTitle{Internships}{\faBriefcase}
% %======================================================================
% \begin{experiences}
%   \experience%
%     [April 2018]%
%     {August 2018}%
%     {Data Engineer @ Leadvisor Technology Inc. (startup company)}%
%     [\begin{itemize}
%       \item Search and scrape product and advertising data from Amazon web
%         (Python, Requests, BeautifulSoup).
%       \item Deployed the Airflow server and database to periodically
%         retrieve product sales and advertising data from Amazon.
%       \item Developed the website (Flask, jQuery) to help customers to
%         optimize their advertising campaigns on Amazon.
%     \end{itemize}]

%   \separator{0.5ex}
%   \experience%
%     [July 2013]%
%     {September 2013}%
%     {Web Developer @ 97 Suifang (startup company)}%
%     [\begin{itemize}
%       \item Developed the back-end (Django) to support user registration,
%         data storage and search.
%       \item Developed the front-end (jQuery, AJAX) to visualize the
%         temporal variations of a patient's examination indicators.
%     \end{itemize}]
% \end{experiences}

\end{document}
